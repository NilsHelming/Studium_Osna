\documentclass[]{article}
\usepackage[utf8]{inputenc}
\usepackage[ngerman]{babel}
\usepackage[T1]{fontenc}
\usepackage{%
	ngerman,
	ae,
	times,  %% hier kann man die Schriftart einstellen
	graphicx,
	url,
	scrlayer-scrpage,
	lastpage,
	mathtools,
	geometry,
	multicol,
	cancel,
	xcolor,
	nicematrix,
	xfrac,
	tikz,
	pgfplots,
	amsmath,
	colortbl,
	centernot,
	dsfont,
	textgreek,
	icomma}
\usepackage[thinlines]{easybmat}
\usetikzlibrary{datavisualization}
\usetikzlibrary{datavisualization.formats.functions}
\usetikzlibrary{intersections}
\pgfplotsset{compat=1.17}
\newcommand{\del}[1]{\cancel{~#1~}}
\NiceMatrixOptions{ last-col,code-for-last-col = \color{blue}\scriptstyle,light-syntax}
\newlength\dlf
\newcommand\alignedhighlight[3]{
  % #1 = color
  % #2 = before alignment
  % #3 = after alignment
  &
  \begingroup
  \settowidth\dlf{$\displaystyle #2$}
  \addtolength\dlf{\fboxsep+\fboxrule}
  \hspace{-\dlf}
  \fcolorbox{#1}{#1}{$\displaystyle #2 #3$}
  \endgroup
}
\newcommand{\reference}[1]{ \text{\small{\textcolor{blue}{(#1)}}} }

\newcommand{\topic}{Elektrotechnisch-phys. Grundlagen}
\newcommand{\subtopic}{Aufgaben 1.1 - 1.29}
\newcommand{\authors}{Nils Helming}

%Head and Footnotes
\setlength{\headheight}{2.1\baselineskip} %baselineskip = minimum distance bbetween the bottom of one line to another.
\geometry{bottom = 3cm}
\setlength{\headsep}{\baselineskip}
\ihead[\topic\hrule]{\topic\hrule}
\chead[\subtopic\\~]{\subtopic\\~}
\ohead[\authors\\~]{\authors\\~}
\ifoot[~]{~}
\cfoot[~]{~}
\ofoot[Seite \thepage~von \pageref{LastPage}]{Seite \thepage~von \pageref{LastPage}}

%Paragraph spacings
\setlength{\parindent}{0em} %em = with of an 'M'
\setlength{\parskip}{1ex} %ex = height of an 'x'


\newcommand{\V}{\lor}
\newcommand{\A}{\land}
\newcommand{\T}[1]{\overline{#1}}
\newcommand{\eq}{\Leftrightarrow}
\newcommand{\rarr}{\Rightarrow}
\newcommand{\red}[1]{\textcolor{red}{#1}}

\newcommand{\unit}[1]{\text{#1}}
\newcommand{\fracunit}[2]{\frac{\unit{#1}}{\unit{#2}}}
\newcommand{\textsq}[1]{\ensuremath{\text{#1}^2}}
\newcommand{\tdot}{\ensuremath{\cdot}}


\begin{document}
\section*{Aufgabe 1.1}
	Der Kamerasensor nimmt pro absorbiertem Lichtteilchen ein Elektron auf.
	Es flie"st ein Strom von 1,6nA über eine Zeit von 0,1\textmu s.

	Wie viele Lichtteilchen wurden absorbiert?

	Elementarladung $e = 1,602*10^{-19}$As

	Damit errechnen wir:
	\begin{align*}
	&&	I &= \frac{Q}{t} &&\\
	\text{bzw.}&&  I &= \frac{n \cdot e}{t} &&\\
	\eq&&  n &= \frac{e}{I \cdot t} &&\\
	\text{durch einsetzen von $e$, $I$ und $t$} && n &= \frac{1,602*10^{-19}\unit{As}}{1,6*10^{-9}\unit{A} \cdot 0,1*10^{-6}\unit{s}}&&\\
	&& &= \frac{1,602*10^{-19}\del{\unit{As}}}{0,16*10^{-15}\del{\unit{As}}} &&\\
	&& &= \frac{1,602}{1600} &&\\
	&& &\approx 1000 &&\\
	\end{align*}
\section*{Aufgabe 1.2}
	Eine Ader hat einen Durchmesser von $d = 1,38\unit{mm}$.\\ Der maximal erlaubte Strom beträgt $I = 19,5\unit{A}$.

	Damit beträgt die Durchschnittsfläche einer Ader $A = \pi \left(\frac{1,38\unit{mm}}{2}\right)^2 \approx 1,5\unit{mm}^2$

	Und damit die maximale Stromdichte\\
	$ J = \frac{I}{A} = \frac{19,5\unit{A}}{1,5\unit{mm}^2} = 13 \frac{\unit{A}}{\text{mm}^2}$

\section*{Aufgabe 1.3}
\subsection*{(a)}
	Der Gegenstand wurde zum Aufschlagszeitpunt für $t=3\unit{s}$ mit $a=9,81 \unit{m}/\unit{s}^2$ beschleunigt. Das bedeutet mit einer Anfangsgeschwindigkeit von $0\unit{m}/\unit{s}$:\\
	$ v = \int a~dt = \int_{t = 0\unit{s}}^{3\unit{s}}9,81\frac{\unit{m}}{\text{s}^2}~dt = 9,81\frac{\unit{m}}{\text{s}^2} \cdot 3\unit{s} = 29,43\fracunit{m}{s}$
\subsection*{(b)}
	Zuzüglich der Anfangsposition von $0\unit{m}$ ergibt sich:\\
	$ d = \int\int a~dt~dt = \int at~dt = \frac{1}{2}at^2 = \frac{1}{2}9,81\frac{\unit{m}}{\text{s}^2} \cdot (3\unit{s})^2 = 44,145\unit{m}$
\subsection*{(c)}
	Für eine Distanz von $d=44,145\unit{m}$ würde Schall mit einer Geschwindigkeit von $v = 343\fracunit{m}{s}$ die so errechnete Zeit benötigen:\\
	$v = \frac{d}{t} \eq t = \frac{d}{v} = \frac{44,145\unit{m}}{343\fracunit{m}{s}} \approx \frac{1}{8}\unit{s}$
\subsection*{(d)}
	Angenommen das Objekt benötigt $\fracunit{23}{8}\unit{s}$ der $3\unit{s}$ um den Grund des Brunnens zu erreichen, können wir die Rechnung aus (b) wiederholen:\\
	$ d = \frac{1}{2}at^2 = \frac{1}{2}9,81\frac{\unit{m}}{\text{s}^2} \cdot (\fracunit{23}{8}\unit{s})^2 \approx 40,5\unit{m}$

\section*{Aufgabe 1.4}
\subsection*{(a)}
	Der PKW fährt mit einer Geschwindigkeit von $100$km/h (ca. $27,8\unit{m/s}$). Somit legt das Fahrzeug $13,9\unit{m}$ innerhalb einer halben Sekunde zurück.
\subsection*{(b)}
	Mit einer Masse von $m = 1500\unit{kg}$ und einer Anfangsgeschwindigkeit von $v_0 = 27,8\unit{m/s}$. Wird der PKW nun mit einer konstanten Kraft $F = 12500\unit{N}$ abgebremst, das bedeutet:\\
	$F = m \cdot a \eq a = \frac{F}{m} = \frac{12500\unit{N}}{1500\unit{kg}} = \frac{25}{3}\fracunit{kg~m}{kg~\textsq{s}} = \frac{25}{3} \fracunit{m}{\textsq{s}}$


	Sei $v(t)$ die Funktion, welche die momentante Geschwindigkeit des PKW während des Bremsprozesses darstellt. Diese ist gegeben durch:\\
	$v(t) = \int -a~dt = -at + v_0 = -\frac{25}{3} \fracunit{m}{\textsq{s}} \cdot t + 27,8\unit{m/s}$\\
	Und $s(t)$ die Funktion, welche die zurückgelegte Strecke seit Beginn des Bremsprozesses angibt:\\
	$s(t)= \int v(t) ~dt = \int -at + v_0~dt = -\frac{1}{2} at^2 + v_0t = -\frac{25}{6}\fracunit{m}{\textsq{s}} \cdot t^2 + 27,8\unit{m/s} \cdot t$


	Gesucht ist der Zeitpunkt zu welchem der PKW $0\unit{m/s}$ erreicht, und die Strecke die zu diesem Zeitpunkt zurückgelegt wurde:
	\begin{flalign*}
		v(t) = 0\unit{m/s} \eq& -\frac{25}{3} \fracunit{m}{\textsq{s}} \cdot t + 27,8\unit{m/s} = 0\unit{m/s} &&\\
		\eq& t = \frac{-27,8\fracunit{m}{s}}{-\frac{25}{3}\fracunit{m}{\textsq{s}}} \approx \frac{10}{3}\unit{s} &&\\
	\end{flalign*}
	$s(\frac{10}{3}\unit{s}) = -\frac{25}{6}\fracunit{m}{\textsq{s}} \cdot (\frac{10}{3}\unit{s})^2 + 27,8\unit{m/s} \cdot \frac{10}{3}\unit{s} = 46,4\unit{m}$

\subsection*{(c)}
	Aus den $13,9\unit{m}$ vor Beginn des Bremsprozesses und den $46,4\unit{m}$ innerhalb dessen ergibt sich eine insgesamt zurückgelegte Strecke von $60,3\unit{m}$.

\subsection*{(d)}
	Mit Änderung der Anfangsgeschwindigkeit auf $v_0 = 130\unit{km/h} \approx 36\unit{m/s}$ legt der PKW in der ersten halben Sekunde also zunächst $18\unit{m}$ zurück und für den Bremsweg ergibt sich:\\
	$v(t) = -\frac{25}{3} \fracunit{m}{\textsq{s}} \cdot t + 36\unit{m/s}$\\
	$s(t) = -\frac{25}{6} \fracunit{m}{\textsq{s}} \cdot t^2 + 36\unit{m/s} \cdot t$\\
	$\Rightarrow t = \frac{-36\fracunit{m}{s}}{-\frac{25}{3}\fracunit{m}{\textsq{s}}} \approx \frac{22}{5}\unit{s}$\\
	$s(\frac{22}{5}\unit{s}) = -\frac{25}{6}\fracunit{m}{\textsq{s}} \cdot (\frac{22}{5}\unit{s})^2 + 36\unit{m/s} \cdot \frac{22}{5}\unit{s} \approx 77,7\unit{m}$\\
	Insgesamt legt der PKW bei einer Anfangsgeschwindigkeit von $130\unit{km/h}$ also $95,7\unit{m}$ zurück.


\section*{Aufgabe 1.5}
	$\rho_{Cu} = 0,0176*10^{-6}\unit{\textOmega m}$\\
	$l = 40\unit{cm} = 0,4\unit{m}$\\
	$A = a \cdot b = 35\unit{\textmu m} \cdot b = 35*10^{-6}\unit{m} \cdot b$\\
	$R = 1\unit{\textOmega} = \frac{\rho \cdot l}{A} = \frac{\rho_{Cu} \cdot l}{a \cdot b}$\\
	$\Rightarrow b = \frac{\rho_{Cu} \cdot l}{a \cdot R} = \frac{0,0176*10^{-6}\unit{\textOmega m} \cdot 0,4\unit{m}}{35*10^{-6}\unit{m} \cdot 1\unit{\textOmega}} = 2,01*10^{-4}\unit{m} = 0,201\unit{mm}$\\

\section*{Aufgabe 1.6}
	$\rho = 10^{12}\unit{\textOmega cm}$\\
	$l = 1\unit{mm} = 10^{-1}\unit{cm}$\\
	$A = 100\unit{\textsq{cm}}= 10^2\unit{\textsq{cm}}$\\
	$R = \frac{\rho \cdot l}{A} = \frac{10^{12}\unit{\textOmega cm} \cdot 10^{-1}\unit{cm}}{10^2\unit{\textsq{cm}}} = 10^{9}\unit{\textOmega}$


\end{document}