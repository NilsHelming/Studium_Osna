\documentclass[]{article}
\usepackage[utf8]{inputenc}
\usepackage[ngerman]{babel}
\usepackage[T1]{fontenc}
\usepackage{%
	ngerman,
	ae,
	times,  %% hier kann man die Schriftart einstellen
	graphicx,
	url,
	scrlayer-scrpage,
	lastpage,
	mathtools,
	geometry,
	multicol,
	cancel,
	xcolor,
	nicematrix,
	xfrac,
	tikz,
	pgfplots,
	amsmath,
	amssymb,
	colortbl,
	centernot,
	dsfont,
	textgreek,
	icomma}
\usepackage[thinlines]{easybmat}
\usetikzlibrary{datavisualization}
\usetikzlibrary{datavisualization.formats.functions}
\usetikzlibrary{intersections}
\pgfplotsset{compat=1.17}
\newcommand{\del}[1]{\cancel{~#1~}}
\NiceMatrixOptions{ last-col,code-for-last-col = \color{blue}\scriptstyle,light-syntax}
\newlength\dlf
\newcommand\alignedhighlight[3]{
  % #1 = color
  % #2 = before alignment
  % #3 = after alignment
  &
  \begingroup
  \settowidth\dlf{$\displaystyle #2$}
  \addtolength\dlf{\fboxsep+\fboxrule}
  \hspace{-\dlf}
  \fcolorbox{#1}{#1}{$\displaystyle #2 #3$}
  \endgroup
}
\newcommand{\reference}[1]{ \text{\small{\textcolor{blue}{(#1)}}} }

\newcommand{\topic}{Mathematik 1 f"ur Informatiker}
\newcommand{\subtopic}{"Ubungsblatt 4}
\newcommand{\authors}{Nils Helming}

%Head and Footnotes
\setlength{\headheight}{2.1\baselineskip} %baselineskip = minimum distance bbetween the bottom of one line to another.
\geometry{bottom = 3cm}
\setlength{\headsep}{\baselineskip}
\ihead[\topic\hrule]{\topic\hrule}
\chead[\subtopic\\~]{\subtopic\\~}
\ohead[\authors\\~]{\authors\\~}
\ifoot[~]{~}
\cfoot[~]{~}
\ofoot[Seite \thepage~von \pageref{LastPage}]{Seite \thepage~von \pageref{LastPage}}

%Paragraph spacings
\setlength{\parindent}{0em} %em = with of an 'M'
\setlength{\parskip}{1ex} %ex = height of an 'x'


\newcommand{\V}{\lor}
\newcommand{\A}{\land}
\newcommand{\T}[1]{\overline{#1}}
\newcommand{\eq}{\Leftrightarrow}
\newcommand{\rarr}{\Rightarrow}
\newcommand{\red}[1]{\textcolor{red}{#1}}

\newcommand{\unit}[1]{\text{#1}}
\newcommand{\fracunit}[2]{\frac{\unit{#1}}{\unit{#2}}}
\newcommand{\textsq}[1]{\ensuremath{\text{#1}^2}}
\newcommand{\textpow}[2]{\ensuremath{\text{#1}^{#2}}}
\newcommand{\tdot}{\ensuremath{\cdot}}


\begin{document}

\section*{1. Aufgabe}
\subsection*{a)}
\begin{align*}
	&& |3r-6| &= r+2&&\\
	&& |3(r-2)| &= r+2&&\\
	&& |3|\cdot|r-2| &= r+2&&\\
	&& 3\cdot|r-2| &= r+2&&\\
\end{align*}
\begin{center}\begin{minipage}{0.45\textwidth}
	1. Fall: $r-2 \geq 0 \eq r \in [2, +\infty)$
	\begin{align*}
		&& 3\cdot|r-2| &= r+2&&\\
		&& 3\cdot(r-2) &= r+2&&\\
		&& 3r-6&= r+2&|-r+6&\\
		&& 2r&= 8&|:2&\\
		&& r&= 4&&\\
		&& \mathbb{L}_1&= \{4\}&&\\
	\end{align*}
\end{minipage} ~\vline~ \begin{minipage}{0.45\textwidth}
	2. Fall: $r-2 < 0 \eq r \in (-\infty,2)$
	\begin{align*}
		&& 3\cdot|r-2| &= r+2&&\\
		&& -3\cdot(r-2) &= r+2&&\\
		&& -3r+6 &= r+2&|-r-6&\\
		&& -4r &= -4&|:(-4)&\\
		&& r &= 1&&\\
		&& \mathbb{L}_2&= \{1\}&&\\
	\end{align*}
\end{minipage}\end{center}
\begin{align*}
	&\rarr& \mathbb{L}&= \mathbb{L}_1 \cup \mathbb{L}_2 = \{4\} \cup \{1\} = \{1,4\} &&\\
\end{align*}

\subsection*{b)}
\begin{align*}
	&& 2 &> |s-3|&&\\
\end{align*}
\begin{center}\begin{minipage}{0.45\textwidth}
	1. Fall: $s-3 \geq 0 \eq s \in [3, +\infty)$
	\begin{align*}
		&& 2 &> |s-3|&&\\
		&& 2 &> s-3&|+3&\\
		&& 5 &> s&&\\
		&& 5 &> s&&\\
		&& \mathbb{L}_1&= [3,5)&&\\
	\end{align*}
\end{minipage} ~\vline~ \begin{minipage}{0.45\textwidth}
	2. Fall: $s-3 < 0 \eq s \in (-\infty,3)$
	\begin{align*}
		&& 2 &> |s-3|&&\\
		&& 2 &> -(s-3)&|\cdot (-1)&\\
		&& -2 &< s-3&|+3&\\
		&& 1 &< s&&\\
		&& \mathbb{L}_2&= (1,3)&&\\
	\end{align*}
\end{minipage}\end{center}
\begin{align*}
	&\rarr& \mathbb{L}&= \mathbb{L}_1 \cup \mathbb{L}_2 = [3,5) \cup (1,3) = (1,5)&&\\
\end{align*}

\subsection*{c)}
\begin{align*}
	&& \frac{1}{|t+1|} &\geq 6 \rarr t \neq -1 & &\\
	\\
	&& \frac{1}{|t+1|} &\geq 6& | \cdot |t+1|&\\
	&& 1 &\geq 6 \cdot |t+1|& |:6&\\
	&& \frac{1}{6} &\geq |t+1|&&\\
\end{align*}
\begin{center}\begin{minipage}{0.45\textwidth}
	1. Fall: $t+1 > 0 \eq t \in (-1,+\infty)$
	\begin{align*}
		&& \frac{1}{6} &\geq |t+1|  &&\\
		&& \frac{1}{6} &\geq t+1  &|-1&\\
		&& -\frac{5}{6} &\geq t  &&\\
		&& \mathbb{L}_1&= (-\infty,-5/6] \cap (-1,+\infty)&&\\
		&& &= (-1,-5/6]&&\\
	\end{align*}
\end{minipage} ~\vline~ \begin{minipage}{0.45\textwidth}
	2. Fall: $t+1 < 0 \eq t \in (-\infty,-1)$
	\begin{align*}
		&& \frac{1}{6} &\geq |t+1|&&\\
		&& \frac{1}{6} &\geq -(t+1)&|\cdot(-1)&\\
		&& -\frac{1}{6} &\leq t+1&|-1&\\
		&& -\frac{7}{6} &\leq t&&\\
		&& \mathbb{L}_2&= [-7/6,+\infty) \cap (-\infty,-1)&&\\
		&& &= [-7/6,-1)&&\\
	\end{align*}
\end{minipage}\end{center}
\begin{align*}
	&\rarr& \mathbb{L} = \mathbb{L}_1 \cup \mathbb{L}_2 &= (-1,-5/6] \cup [-7/6,-1) &&\\
	&& &= [-7/6,-1) \cup (-1,-5/6] &&\\
	&& &= [-7/6,-5/6] ~ \backslash ~ \{-1\} &&\\
\end{align*}

\section*{2. Aufgabe}
\subsection*{a)}
	\begin{align*}
		&& a \cdot b &= 0 &&\\
		&\text{Existenz neutraler Elemente:}& &= 0 + 0 \cdot a &&\\
	\end{align*}
\section*{3. Aufgabe}
\subsection*{(1)}
	$\star$ ist kommutativ, denn hier ist in der Verknüpfungstabelle eine Spiegelsymmetrie über die Diagonale zu beobachten.\\
	$\circ$ hingegen ist nicht kommutativ, denn z.B. $a \circ b \neq b \circ a$.

\section*{4. Aufgabe}
\subsection*{(1)}
	Zu zeigen ist: a) $\forall a \in \mathbb{N}_0: a \circ 0 = a$ und b) $\forall a \in \mathbb{N}_0: a \circ a = 0$\\
	a)
	\begin{align*}
		&& a \circ 0 &= |a-0| &&\\
		&& &= |a|-|0| &&\\
		&\text{da $a\in\mathbb{N}_0$}& &= a-0 &&\\
		&& &= a &&\\
	\end{align*}
	b)
	\begin{align*}
		&& a \circ a &= |a-a| &&\\
		&& &= |0| &&\\
		&& &= 0 &&\\
	\end{align*}

\section*{5. Aufgabe}
\subsection*{a)}
Assoziativgesetz der Addition:
\begin{align*}
	&& \T{1} + (\T{2} + \T{4}) &= \T{1} + \T{1} &&\\
	&& &= \T{2} &&\\
	&& (\T{1} + \T{2}) + \T{4} &= \T{3} + \T{4} &&\\
	&& &= \T{2} &&\\
	&\rarr& \T{1} + (\T{2} + \T{4}) &= (\T{1} + \T{2}) + \T{4} &&\\
\end{align*}
Kommutativgesetz der Multiplikation:
\begin{align*}
	&& \T{2} \bullet \T{4} &= \T{3} &&\\
	&& \T{4} \bullet \T{2} &= \T{3} &&\\
	&\rarr& \T{2} \bullet \T{4} &= \T{4} \bullet \T{2} &&\\
\end{align*}
Distributivgesetz:
\begin{align*}
	&& \T{2} \bullet (\T{3} + \T{4}) &= \T{2} \bullet \T{2} &&\\
	&& &= \T{4} &&\\
	&& (\T{2} \bullet \T{3}) + (\T{2} \bullet \T{4}) &= \T{1} + \T{3} &&\\
	&& &= \T{4} &&\\
	&\rarr& \T{2} \bullet (\T{3} + \T{4}) &= (\T{2} \bullet \T{3}) + (\T{2} \bullet \T{4}) &&\\
\end{align*}

\section*{6. Aufgabe}
\subsection*{a)}
	\begin{align*}
		&& (\T{12} + \T{9})^2 &= (\T{12+9})^2 &&\\
		&& &= (\T{21})^2 &&\\
		&& &= (\T{4})^2 &&\\
		&& &= \T{4^2} &&\\
		&& &= \T{16} &&\\
		\\
		&& \T{12}^2 + \T{2} \cdot \T{12} \cdot \T{9} + \T{9}^2  &= \T{12^2} + \T{2 \cdot 9} \cdot \T{12} + \T{9^2} &&\\
		&& &= \T{144} + \T{18} \cdot \T{12}  + \T{81} &&\\
		&& &= \T{8} + \T{1} \cdot \T{12}  + \T{13} &&\\
		&& &= \T{8} + \T{12}  + \T{13} &&\\
		&& &= \T{8+12+13} &&\\
		&& &= \T{33} &&\\
		&& &= \T{16} &&\\
		\\
		&\rarr& (\T{12} + \T{9})^2 &= \T{12}^2 + \T{2} \cdot \T{12} \cdot \T{9} + \T{9}^2 &&\\
	\end{align*}

\end{document}