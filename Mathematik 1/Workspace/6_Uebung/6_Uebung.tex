\documentclass[]{article}
\usepackage[utf8]{inputenc}
\usepackage[ngerman]{babel}
\usepackage[T1]{fontenc}
\usepackage{%
	ngerman,
	ae,
	times,  %% hier kann man die Schriftart einstellen
	graphicx,
	url,
	scrlayer-scrpage,
	lastpage,
	mathtools,
	geometry,
	multicol,
	cancel,
	xcolor,
	nicematrix,
	xfrac,
	tikz,
	pgfplots,
	amsmath,
	amssymb,
	colortbl,
	centernot,
	dsfont,
	textgreek,
	icomma}
\usepackage[thinlines]{easybmat}
\usetikzlibrary{datavisualization}
\usetikzlibrary{datavisualization.formats.functions}
\usetikzlibrary{intersections}
\pgfplotsset{compat=1.17}
\newcommand{\del}[1]{\cancel{~#1~}}
\NiceMatrixOptions{ last-col,code-for-last-col = \color{blue}\scriptstyle,light-syntax}
\newlength\dlf
\newcommand\alignedhighlight[3]{
  % #1 = color
  % #2 = before alignment
  % #3 = after alignment
  &
  \begingroup
  \settowidth\dlf{$\displaystyle #2$}
  \addtolength\dlf{\fboxsep+\fboxrule}
  \hspace{-\dlf}
  \fcolorbox{#1}{#1}{$\displaystyle #2 #3$}
  \endgroup
}
\newcommand{\reference}[1]{ \text{\small{\textcolor{blue}{(#1)}}} }

\newcommand{\topic}{Mathematik 1 f"ur Informatiker}
\newcommand{\subtopic}{"Ubungsblatt 6}
\newcommand{\authors}{Nils Helming}

%Head and Footnotes
\setlength{\headheight}{2.1\baselineskip} %baselineskip = minimum distance bbetween the bottom of one line to another.
\geometry{bottom = 3cm}
\setlength{\headsep}{\baselineskip}
\ihead[\topic\hrule]{\topic\hrule}
\chead[\subtopic\\~]{\subtopic\\~}
\ohead[\authors\\~]{\authors\\~}
\ifoot[~]{~}
\cfoot[~]{~}
\ofoot[Seite \thepage~von \pageref{LastPage}]{Seite \thepage~von \pageref{LastPage}}

%Paragraph spacings
\setlength{\parindent}{0em} %em = with of an 'M'
\setlength{\parskip}{1ex} %ex = height of an 'x'


\newcommand{\V}{\lor}
\newcommand{\A}{\land}
\newcommand{\T}[1]{\overline{#1}}
\newcommand{\IT}[1]{\underline{\underline{#1}}}
\newcommand{\eq}{\Leftrightarrow}
\newcommand{\rarr}{\Rightarrow}
\newcommand{\red}[1]{\textcolor{red}{#1}}

\newcommand{\unit}[1]{\text{#1}}
\newcommand{\fracunit}[2]{\frac{\unit{#1}}{\unit{#2}}}
\newcommand{\textsq}[1]{\ensuremath{\text{#1}^2}}
\newcommand{\textpow}[2]{\ensuremath{\text{#1}^{#2}}}
\newcommand{\tdot}{\ensuremath{\cdot}}


\begin{document}
\begin{center}$\begin{BMAT}()[5pt,2cm,2cm]{lccc|cl}{c|ccc|ccc|ccc|ccc}
	& x & y & z &   &\\
	%
	\red{\text{I:}}  & 3 & 1 & 1 & b & \red{\leftarrow \text{II}}\\
	\red{\text{II:}} &-1 & 1 & 1 & 2 & \red{\leftarrow \text{III}}\\
	\red{\text{III:}}& 1 & 3 & 3 &-2 & \red{\leftarrow \text{I}}\\
	%
	\red{\text{I:}}  &-1 & 1 & 1 & 2 & \red{}\\
	\red{\text{II:}} & 1 & 3 & 3 &-2 & \red{\leftarrow +\text{I}}\\
	\red{\text{III:}}& 3 & 1 & 1 & b & \red{\leftarrow +3\text{I}}\\
	%
	\red{\text{I:}}  &-1 & 1 & 1 & 2 & \red{}\\
	\red{\text{II:}} & 0 & 4 & 4 & 0 & \red{}\\
	\red{\text{III:}}& 0 & 4 & 4 &b+6& \red{\leftarrow -\text{II}}\\
	%
	\red{\text{I:}}  &-1 & 1 & 1 & 2 & \red{}\\
	\red{\text{II:}} & 0 & 4 & 4 & 0 & \red{}\\
	\red{\text{III:}}& 0 & 0 & 0 &b+6& \red{}\\
\end{BMAT}$\end{center}

$\begin{pmatrix}
	1&2&3\\2&3&4
\end{pmatrix}$
\newpage
\section*{6. Aufgabe}
\subsection*{a)}

$\vec{a}  = \begin{pmatrix}
	-1\\1\\3
\end{pmatrix}$\\
$\vec{b}  = \begin{pmatrix}
	2\\-1\\0
\end{pmatrix}$\\
$\vec{d}  = \begin{pmatrix}
	1\\2\\3
\end{pmatrix}$\\

$\vec{c} = \vec{a} \times \vec{b} = \begin{pmatrix}-1\\1\\3\end{pmatrix} \times \begin{pmatrix}2\\-1\\0\end{pmatrix} = \begin{pmatrix}3\\6\\-1\end{pmatrix}$

\begin{center}$\begin{BMAT}()[5pt,2cm,2cm]{lccc|cl}{c|ccc|ccc|ccc|ccc}
	& \alpha & \beta & \gamma &   &\\
	%
	\red{\text{I:}}  & -1 & 2 & 3 & 1 & \red{}\\
	\red{\text{II:}} &1 & -1 & 6 & 2 & \red{\leftarrow +\text{I}}\\
	\red{\text{III:}}& 3 & 0 & -1 &3 & \red{\leftarrow +(3\cdot\text{I})}\\
	%
	\red{\text{I:}}  &-1 & 2 & 3 & 1 & \red{}\\
	\red{\text{II:}} & 0 & 1 & 9 & 3 & \red{}\\
	\red{\text{III:}}& 0 & 6 & 8 & 6 & \red{\leftarrow -(6\cdot\text{II})}\\
	%
	\red{\text{I:}}  &-1 & 2 & 3 & 1 & \red{}\\
	\red{\text{II:}} & 0 & 1 & 9 & 3 & \red{}\\
	\red{\text{III:}}& 0 & 0 &-46&-12& \red{\leftarrow :(-2)}\\
	%
	\red{\text{I:}}  &-1 & 2 & 3 & 1 & \red{}\\
	\red{\text{II:}} & 0 & 1 & 9 & 3 & \red{}\\
	\red{\text{III:}}& 0 & 0 & 23& 6 & \red{}\\
\end{BMAT}$\end{center}
\begin{align*}
	\text{\red{III}:}&& 0\alpha + 0\beta + 23\gamma &= 6 &&\\
	\eq&& \gamma &= \frac{6}{23} &&\\
	\\
	\text{\red{II}:}&& 0\alpha + 1\beta + 9\gamma &= 3 &&\\
	\eq&& \beta + \frac{9 \cdot 6}{23} &= 3 &&\\
	\eq&& \beta &= \frac{69}{23} - \frac{54}{23} &&\\
	\eq&& \beta &= \frac{15}{23}&&\\
	\\
	\text{\red{I}:}&& -1\alpha + 2\beta + 3\gamma &= 1 &&\\
	\eq&& -\alpha + \frac{2 \cdot 15}{23} + \frac{3 \cdot 6}{23} &= 1&&\\
	\eq&& -\alpha &= \frac{23}{23} -\frac{30}{23} - \frac{18}{23}&&\\
	\eq&& -\alpha &= -\frac{25}{23}&&\\
	\eq&& \alpha &= \frac{25}{23}&&\\
\end{align*}
\subsection*{b)}
$\vec{a}  = \begin{pmatrix}
	-1\\1\\3
\end{pmatrix}$\\
$\vec{b}  = \begin{pmatrix}
	0\\4\\1
\end{pmatrix}$\\
$\vec{c}  = \begin{pmatrix}
	\alpha\\2\\3
\end{pmatrix}$\\
\newpage
\begin{align*}
	&& \begin{pmatrix}0\\0\\0\end{pmatrix} &= t_1 \cdot \vec{a} + t_2 \cdot \vec{b}+ t_3 \cdot \vec{c} &&\\
	&&  &= t_1\cdot \begin{pmatrix}-1\\1\\3
	\end{pmatrix} + t_2 \cdot \begin{pmatrix}0\\4\\1\end{pmatrix}+ t_3 \cdot \begin{pmatrix}\alpha\\2\\3\end{pmatrix} &&\\
\end{align*}

\begin{center}$\begin{BMAT}()[5pt,2cm,2cm]{lccc|cl}{c|ccc|ccc|ccc|ccc}
	& t_1 & t_2 & t_3 &   &\\
	%
	\red{\text{I:}}  &-1 & 0 & \alpha & 0 & \red{\leftarrow \text{II}}\\
	\red{\text{II:}} & 1 & 4 & 2 & 0 & \red{\leftarrow \text{III}}\\
	\red{\text{III:}}& 3 & 1 & 3 & 0 & \red{\leftarrow \text{I}}\\
	%
	\red{\text{I:}} & 1 & 4 & 2 & 0 & \red{}\\
	\red{\text{II:}}& 3 & 1 & 3 & 0 & \red{\leftarrow -(3\cdot\text{I})}\\
	\red{\text{III:}}  &-1 & 0 & \alpha & 0 & \red{\leftarrow +\text{I}}\\
	%
	\red{\text{I:}}  & 1 & 4 & 2 & 0 & \red{}\\
	\red{\text{II:}} & 0 &-11& -3 & 0 & \red{}\\
	\red{\text{III:}}& 0 & 4 & \alpha+2 & 0 & \red{\leftarrow (11 \cdot \text{III}) + (4 \cdot \text{II})}\\
	%
	\red{\text{I:}}  & 1 & 4 & 2 & 0 & \red{}\\
	\red{\text{II:}} & 0 &-11& -3 & 0 & \red{}\\
	\red{\text{III:}}& 0 & 0 & 11\alpha+10 & 0 & \red{}\\
\end{BMAT}$\end{center}

Falls $11\alpha +10 \neq 0$ gilt, hätte dieses LGS eine einzige Lösung, nämlich $t_1=0, t_2 = 0, t_3 = 0$. Also wären die die Vektoren in diesem Fall linear unabhängig.

Im Gegensatz, falls $11\alpha +10 = 0$, wären die Vektoren also linear abhängig. Also sind für $\alpha = -10/11$ die Vektoren linear abhängig.

Probe:
\begin{align*}
	&& \begin{pmatrix}\frac{-10}{11}\\2\\3\end{pmatrix} &= t_1\cdot \begin{pmatrix}-1\\1\\3
	\end{pmatrix} + t_2 \cdot \begin{pmatrix}0\\4\\1\end{pmatrix}  &&\\
	&& \frac{-10}{11}&= -1t_1 &&\\
	&& \frac{10}{11}&= t_1 &&\\
	\\
	&& 2&= 1t_1 + 4t_2 &&\\
	&& \frac{12}{11}&= 4t_2 &&\\
	&& \frac{3}{11}&= t_2 &&\\
	\\
	&& 3&= 3t_1 + 1t_2 &&\\
	&& &= 3\frac{10}{11} + \frac{3}{11} &&\\
	&& &= \frac{30}{11} + \frac{3}{11} &&\\
	&& &= \frac{33}{11} &&\\
\end{align*}
\newpage
\subsection*{c)}
$\vec{a}  = \begin{pmatrix}
	-1\\1\\3
\end{pmatrix}$\\
$\vec{b}  = \begin{pmatrix}
	0\\4\\1
\end{pmatrix}$\\
$\vec{c}  = \begin{pmatrix}
	1\\2\\3
\end{pmatrix}$\\
Aus b) wissen wir, dass $\{\vec{a},\vec{b},\vec{c}\}$ linear unabhängig sind, denn $1 \neq \frac{-10}{11}$

$\vec{v} = \begin{pmatrix}
	v_1\\v_2\\v_3
\end{pmatrix} \in \mathbb{R}^3$

$\vec{v} = \alpha\cdot\vec{a}+\beta\cdot\vec{b}+\gamma\cdot\vec{c}$

\begin{center}$\begin{BMAT}()[5pt,2cm,2cm]{lccc|cl}{c|ccc}
	& \alpha & \beta & \gamma &   &\\
	%
	\red{\text{I:}}  &-1 & 0 & 1 & v_1 & \red{}\\
	\red{\text{II:}} & 1 & 4 & 2 & v_2 & \red{}\\
	\red{\text{III:}}& 3 & 1 & 3 & v_3 & \red{}\\
\end{BMAT}$\end{center}

% $\IT{A}$

% \begin{align*}
% 	\text{\red{II}:}&& 0x + 4y + 4z &= 0 &&\\
% 	\eq&& 4y &= -4z &&\\
% 	&&  &= -4t &&\\
% 	\rarr&&  y&= -t &&\\
% 	\\
% 	\text{\red{I}:}&& -1x + 1y + 1z &= 2 &&\\
% 	\eq&& 1x &= 1y + 1z -2 &&\\
% 	\eq&& 1x &= 1y + 1z -2 &&\\
% \end{align*}

\end{document}
