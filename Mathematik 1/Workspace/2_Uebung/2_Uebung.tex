\documentclass[]{article}
\usepackage[utf8]{inputenc}
\usepackage[ngerman]{babel}
\usepackage[T1]{fontenc}
\usepackage{%
	ngerman,
	ae,
	times,  %% hier kann man die Schriftart einstellen
	graphicx,
	url,
	scrlayer-scrpage,
	lastpage,
	mathtools,
	geometry,
	multicol,
	cancel,
	xcolor,
	nicematrix,
	xfrac,
	tikz,
	pgfplots,
	amsmath,
	amssymb,
	colortbl,
	centernot,
	dsfont,
	textgreek,
	icomma}
\usepackage[thinlines]{easybmat}
\usetikzlibrary{datavisualization}
\usetikzlibrary{datavisualization.formats.functions}
\usetikzlibrary{intersections}
\pgfplotsset{compat=1.17}
\newcommand{\del}[1]{\cancel{~#1~}}
\NiceMatrixOptions{ last-col,code-for-last-col = \color{blue}\scriptstyle,light-syntax}
\newlength\dlf
\newcommand\alignedhighlight[3]{
  % #1 = color
  % #2 = before alignment
  % #3 = after alignment
  &
  \begingroup
  \settowidth\dlf{$\displaystyle #2$}
  \addtolength\dlf{\fboxsep+\fboxrule}
  \hspace{-\dlf}
  \fcolorbox{#1}{#1}{$\displaystyle #2 #3$}
  \endgroup
}
\newcommand{\reference}[1]{ \text{\small{\textcolor{blue}{(#1)}}} }

\newcommand{\topic}{Mathematik 1 f"ur Informatiker}
\newcommand{\subtopic}{"Ubungsblatt 2}
\newcommand{\authors}{Nils Helming}

%Head and Footnotes
\setlength{\headheight}{2.1\baselineskip} %baselineskip = minimum distance bbetween the bottom of one line to another.
\geometry{bottom = 3cm}
\setlength{\headsep}{\baselineskip}
\ihead[\topic\hrule]{\topic\hrule}
\chead[\subtopic\\~]{\subtopic\\~}
\ohead[\authors\\~]{\authors\\~}
\ifoot[~]{~}
\cfoot[~]{~}
\ofoot[Seite \thepage~von \pageref{LastPage}]{Seite \thepage~von \pageref{LastPage}}

%Paragraph spacings
\setlength{\parindent}{0em} %em = with of an 'M'
\setlength{\parskip}{1ex} %ex = height of an 'x'


\newcommand{\V}{\lor}
\newcommand{\A}{\land}
\newcommand{\T}[1]{\overline{#1}}
\newcommand{\eq}{\Leftrightarrow}
\newcommand{\rarr}{\Rightarrow}
\newcommand{\red}[1]{\textcolor{red}{#1}}

\newcommand{\unit}[1]{\text{#1}}
\newcommand{\fracunit}[2]{\frac{\unit{#1}}{\unit{#2}}}
\newcommand{\textsq}[1]{\ensuremath{\text{#1}^2}}
\newcommand{\textpow}[2]{\ensuremath{\text{#1}^{#2}}}
\newcommand{\tdot}{\ensuremath{\cdot}}


\begin{document}
\section*{1. Aufgabe}
\subsection*{(a)}
	R ist nicht reflexiv, da es ein Element $1 \in A$ gibt, für welches $(1,1) \notin R$ gilt.

	R ist nicht symmetrisch, denn $(3,6) \in R$ aber $(6,3) \notin R$.

	R ist nicht transitiv, denn $(5,3) \in R \A (3,4) \in R$, aber $(5,4) \notin R$.

	R ist nicht antisymmetrisch, denn $(1,2) \in R \A (2,1) \in R$, obwohl $1 \neq 2$.

\subsection*{(b)}
	Eine Äquivalenzrelation ist reflexiv, transitiv und symmetrisch. Für die Reflexivität müssen wir die Elemente $(1,1)$, $(4,4)$ und $(5,5)$ hinzufügen. Für die Transitivität fehlen zusätzlich $(4,6)$ und $(5,4)$. Und für die Symmetrie fügen wir noch $(6,3)$, $(5,4)$ und $(6,5)$ hinzu. (Durch das Hinzufügen der letzten drei Paare sind keine neuen Konflikte entstanden)

	Zu bestimmen sind nun diese Äquivalenzklassen:
	$[3] = \{3, 4, 5, 6\}$ und $[6] = \{3,5,6\}$
\section*{2. Aufgabe}
\subsection*{(a)}
	Die Relation F ist eine Äquivalenzrelation, denn sie ist...\\
	...reflexiv, da jeder Mensch in der selben Familie ist, wie er selbst.\\
	...transitiv, denn wenn ein Mensch A in der selben Familie ist, wie B und dieser wiederrum in der selben Familie, wie C ist. Dann sind A und C auch in derselben Familie.\\
	...symmetrisch, denn wenn A mit B in der selben Familie sind, dann ist auch B mit A in derselben Familie.\\
	Die Äquivalenzklassen sind hier also die Familien.
\subsection*{(b)}
	Die Relation B ist keine Äquivalenzrelation, denn sie ist weder reflexiv noch symmetrisch. (z.B. Frau Mustermann hat einen Bruder. Sie ist aber nicht ihr eigener Bruder, noch ist sie der Bruder ihres Bruders.)
\subsection*{(c)}
	Die Relation A ist eine Äquivalenzrelation, denn sie ist...\\
	...reflexiv, da jeder Mensch das selbe Alter hat, wie man selbst.\\
	...transitiv, denn wenn ein Mensch A genauso alt ist, wie B und dieser wiederrum genauso alt, wie C ist. Dann sind A und C auch genauso alt.\\
	...symmetrisch, denn wenn A mit B ein identishces Alter haben, dann muss auch B dasselbe Alter, wie A haben.\\
	Die Äquivalenzklassen sind hier also die Altersgruppen/Jahrgänge?.

\section*{3. Aufgabe}
\subsection*{(a)}
	$R_4 = \{(1,1);(1,2);(1,3);(1,4);(1,5)\}$\\
	$R_5 = \{(a,a); (a,b); (a,c)\}$\\
	$R_6 = \{\} = \emptyset$\\
	$R_7 = \{(a,b); (a,c)\}$
\subsection*{(b)}
	$R_1$ ist eine Funktion, denn zu jedem Wert im Definitionsbereich wird durch die Relation genau ein Wert aus dem Wertebereich zugewiesen.\\

	$R_2$ ist keine Funktion, denn nicht jeder Wert im Definitionsbereich steht in relation mit einem Wert aus dem Wertebereich und $a \in B$ steht in relation mit mehreren Werten aus dem Wertebereich.\\

	$R_3$ ist keine Funktion, denn nicht jeder Wert im Definitionsbereich steht in relation mit einem Wert aus dem Wertebereich.\\

	$R_4$ ist keine Funktion, denn nicht jeder Wert im Definitionsbereich steht in relation mit einem Wert aus dem Wertebereich und $1 \in A$ steht in relation mit mehreren Werten aus dem Wertebereich.\\

	$R_5$ ist keine Funktion, denn nicht jeder Wert im Definitionsbereich steht in relation mit einem Wert aus dem Wertebereich und $a \in B$ steht in relation mit mehreren Werten aus dem Wertebereich.\\

	$R_6$ ist keine Funktion, denn nicht jeder Wert im Definitionsbereich steht in relation mit einem Wert aus dem Wertebereich.\\

	$R_7$ ist keine Funktion, denn nicht jeder Wert im Definitionsbereich steht in relation mit einem Wert aus dem Wertebereich.\\

\section*{4. Aufgabe}
\subsection*{(a)}
	$T$ ist reflexiv, denn jedes $x \in M$ Teilt sich selbst.\\

	$T$ ist transitiv, da bei gegebenen $(x,y) \in T$ und $(y,z) \in T$ gilt:
	$\exists k \in \mathbb{Z}: k \cdot x = y$ und $\exists l \in \mathbb{Z}: l \cdot y = z$ aber daraus folgt: $z = l \cdot y = l \cdot (k \cdot x) = (l \cdot k) \cdot x \rarr (x,z) \in T$ da $n = (l \cdot k) \in \mathbb{Z}$.\\

	$T$ ist nicht symmetrisch, denn z.B. $(1,2) \in T$ aber $(2,1) \notin T$.\\

	$T$ ist antisymmetrisch, da aus $(x,y) \in T$ und $(y,x) \in T$ folgt: $\exists k \in \mathbb{Z}: k \cdot x = y$ und $\exists l \in \mathbb{Z}: l \cdot y = x$. Also $x = l \cdot y = l \cdot (k \cdot x) = (l \cdot k) \cdot x \rarr 1 = l \cdot k \rarr k = \frac{1}{l}$ Was mit $k \in \mathbb{Z}$ nur für $k = l = 1$ gelten kann. Damit ist $y = k \cdot x = x$

	Also ist $T$ eine Ordnungsrelation, aber keine Äquivalenzrelation.
\subsection*{(b)}


\section*{5. Aufgabe}
\newpage
\section*{6. Aufgabe}
\subsection*{(a) $\forall n \in \mathbb{N} \text{ gilt: } 11|10^{2n} -1 $}
Behauptung:
\begin{align*}
	 && 11&|10^{2 \cdot n}-1 &&\\
	 \eq&& \forall n \in \mathbb{N} \exists m \in \mathbb{N} &\text{ mit } 10^{2n}-1 = 11\cdot m  &&\\
\end{align*}
Induktionsanfang (Anker): ($n=1$)\\
\begin{align*}
	 && 10^{2 \cdot 1}-1 &= 100 - 1 &&\\
	 && &= 99 &&\\
	 && &= 11 \cdot 9 &&\\
	 && &\rarr m = 9 &&\\
\end{align*}

Induktionsschritt: ($k~ -> k+1$)\\
Induktionsvorraussetzung: Behauptung gilt für $n=k$\\
$11|10^{2\cdot k} - 1$ d.h. \\
$\exists i \in \mathbb{N} \text{ mit } 10^{2\cdot k}-1 = 11\cdot i$

Induktionsbehauptung: Behauptung gilt für $n = k+1$\\
$11|10^{2\cdot (k+1)} - 1$ d.h. \\
$\exists j \in \mathbb{N} \text{ mit } 10^{2\cdot (k+1)}-1 = 11\cdot j$


Beweis: (Zeige Induktionsbehauptung unter verwendung der Induktionsvorraussetzung)
\begin{align*}
	&& 10^{2\cdot (k+1)}-1 &= 10^{2k +2}-1 &&\\
	&&  &= 10^{2k +2}-1 &&\\
	&&  &= (10^{2k} \cdot 10^2) -1 &&\\
	&&  &= ((10^{2k} -1+1)\cdot 10^2) -1 &&\\
	&&  &= (((10^{2k} -1)+1)\cdot 10^2) -1 &&\\
	&&  &= (((11 \cdot i)+1)\cdot 10^2) -1 &&\\
	&&  &= ((11i+1)\cdot 10^2) -1 &&\\
	&&  &= (11 \cdot 100i+100) -1 &&\\
	&&  &= 11 \cdot 100i+99 &&\\
	&&  &= 11 \cdot 100i+11 \cdot 9 &&\\
	&&  &= 11(100i+ 9)&&\\
	\text{mit $j:=100i+9$}&&  &= 11 \cdot j&&\\
\end{align*}

\newpage
\subsection*{(c) $\forall n \in \mathbb{N} \text{ gilt: } \sum_{i=1}^{n}(2\cdot i-1) = n^2 $}
Behauptung:
\begin{align*}
	 && \forall n \in \mathbb{N} \text{ gilt: } \sum_{i=1}^{n}(2\cdot i-1) &= n^2 &&\\
\end{align*}

Induktionsanfang: (Anker) ($n = 1$)
\begin{align*}
	 && \sum_{i=1}^{1}(2\cdot i-1) &= 2 \cdot 1 - 1 &&\\
	 && &= 1 &&\\
	 && &= 1^2 &&\\
\end{align*}

Induktionsschritt: ($k~ -> k+1$)\\
Induktionsvorraussetzung: Behauptung gilt für $n = k$\\
\begin{align*}
	 && \sum_{i=1}^{k}(2\cdot i-1) &= k^2 &&\\
\end{align*}


Induktionsbehauptung: Behauptung gilt für $n = k+1$\\
\begin{align*}
	 && \sum_{i=1}^{k+1}(2\cdot i-1) &= (k+1)^2 &&\\
\end{align*}
Beweis: (Zeige Induktionsbehauptung unter verwendung der Induktionsvorraussetzung)\\
\begin{align*}
	 && \sum_{i=1}^{k+1}(2\cdot i-1) &= (2\cdot 1 -1) + (2\cdot 2 -1)+ (2\cdot 3 -1) + \dots +(2\cdot k -1) +(2\cdot (k+1) -1)&&\\
	 && &= \sum_{i=1}^{k}(2\cdot i-1) +(2\cdot (k+1) -1)&&\\
	 && &= k^2 +(2\cdot (k+1) -1)&&\\
	 && &= k^2 +(2k+2 -1)&&\\
	 && &= k^2 +(2k+1)&&\\
	 && &= k^2 +2k+1&&\\
	 && &= (k+1)^2&&\\
\end{align*}

\newpage
\subsection*{(d) $\forall n \in \mathbb{N}, n \geq 4 \text{ gilt: } 2^n \geq n^2$}
Behauptung:
\begin{align*}
	 && \forall n \in \mathbb{N}, n \geq 4 \text{ gilt: } 2^n &\geq n^2 &&\\
\end{align*}

Induktionsanfang (Anker): ($n=4$)
\begin{align*}
	 && 2^4 &= 16 &&\\
	 &&  &= 4^2 &&\\
	 &&  &\geq 4^2 &&\\
\end{align*}

Induktionsschritt: ($k~ -> k+1$)\\
Induktionsvorraussetzung: Behauptung gilt für $n=k$\\
\begin{align*}
	 && 2^k &\geq k^2 &&\\
\end{align*}

Induktionsbehauptung: Behauptung gilt für $n=k+1$\\
\begin{align*}
	 && 2^{k+1} &\geq (k+1)^2 &&\\
\end{align*}

Beweis: (Zeige Induktionsbehauptung unter verwendung der Induktionsvorraussetzung)\\
\begin{align*}
	 && 2^{k+1} &= 2^k \cdot 2^1 &&\\
	 && &\geq k^2 \cdot 2^1 &&\\
\end{align*}





\end{document}